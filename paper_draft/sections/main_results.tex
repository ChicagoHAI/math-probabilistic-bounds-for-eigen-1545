\section{Main Results}\label{sec:main_results}

We now present our main theoretical contributions. Our first result provides a sharp bound on the spectral norm of random matrices with independent sub-gaussian entries.

\begin{theorem}\label{thm:main}
Let $X$ be an $n \times n$ random matrix with independent sub-gaussian entries with parameter $K$. Then for any $t \geq 0$:
\[ \Prob(\spectral{X} \geq 2K\sqrt{n} + t) \leq 2\exp(-ct^2/K^2) \]
where $c > 0$ is an absolute constant.
\end{theorem}

\begin{proof}
The proof proceeds in three steps:
\begin{enumerate}
    \item First, we decompose $X = Y + Z$ where $Y$ contains the truncated entries and $Z$ the tails
    \item Apply the Matrix Bernstein inequality to $Y$
    \item Control the norm of $Z$ using sub-gaussian tail bounds
\end{enumerate}
The result follows by optimizing the truncation level and combining the bounds.
\end{proof}

Our second main result concerns individual eigenvalues:

\begin{theorem}\label{thm:individual}
Under the same conditions as Theorem \ref{thm:main}, for any $1 \leq k \leq n$ and $t \geq 0$:
\[ \Prob(|\lambda_k(X) - \E[\lambda_k(X)]| \geq t) \leq 4\exp(-ct^2/K^2) \]
\end{theorem}

\begin{proof}
We use Weyl's inequality combined with a net argument on the unit sphere. The full proof is technical and deferred to the appendix.
\end{proof}

Finally, we establish a lower bound showing our results are tight:

\begin{theorem}\label{thm:lower}
There exists a family of $n \times n$ random matrices with sub-gaussian entries such that
\[ \Prob(\spectral{X} \leq 2K\sqrt{n} - t) \leq 2\exp(-c't^2/K^2) \]
for some absolute constant $c' > 0$.
\end{theorem}

\begin{example}
Consider the case where $X_{ij}$ are independent standard Gaussian entries. Numerical experiments confirm that the eigenvalues concentrate around $2\sqrt{n}$ with sub-gaussian tails, matching our theoretical predictions.
\end{example}